\documentclass[twoside,12pt]{article} 

\RequirePackage[latin1]{inputenc}   % Pour les accents
\RequirePackage[frenchb]{babel}      % Vocabulaire en fran^^e7ais
\frenchbsetup{StandardLists}        % 2014 pour \'eviter un conflit avec paralist
\RequirePackage[T1]{fontenc}        % c\'esure de mots accentu\'es

\usepackage{amsmath}
\usepackage{amssymb}

\usepackage{xparse}

\let\si\undefined
\usepackage{siunitx}
\sisetup{input-ignore=~\Mref, output-decimal-marker = {,},
  input-protect-tokens = ~\approx\dots\ge\geq\gg\le\leq\ll\mp\pi\pm\sim}

\usepackage{nopageno}

\NewDocumentCommand\colmat{mmgg}{
	\ensuremath{\left[ \begin{array}[pos]{c} #1 \\ #2 \IfNoValueTF{#3}{}{\\ #3} \IfNoValueTF{#4}{}{\\ #4} \end{array} \right]}
}

\setlength{\parskip}{1ex plus 0.5ex minus 0.2ex}

\newcommand{\spreadtext}[3]{\par\noindent\parbox{0.25\textwidth}{\raggedright#1}%
                                     \parbox{0.25\textwidth}{\centering#2}%
                                     \parbox{0.5\textwidth}{\raggedleft#3}\par}

\begin{document}

\spreadtext{201-NYC-05}{~}{Alg�bre lin�aire}
\begin{center}{{\bfseries\large Devoir 1 : Vecteurs alg�briques}}\end{center}


\vspace{1em}

Un skieur de \SI{90}{kg} glisse sur une pente perpendiculaire au vecteur normal $\vec{n} = \colmat{2}{11}$. La force de friction totale sur le skieur (due � l'air et � la neige sous les skis) est de \SI{30}{N}. \textit{Gardez 4 d�cimales dans vos r�ponses.}
\begin{enumerate}
\item Sachant que $g\approx\SI{9.8}{m/s^2}$, donnez le vecteur $\vec{F}_g$ qui repr�sente la force de gravit� sur le skieur.
\item � l'aide d'une projection, trouvez le vecteur qui repr�sente la partie de la force de gravit� qui est perpendiculaire � la pente. Donnez ensuite le vecteur $\vec{F}_N$ qui repr�sente la force normale qu'exerce la pente sur le skieur.
\item Trouvez un vecteur parall�le � la pente.
\item Trouvez le vecteur $\vec{F}_f$ qui repr�sente la force de friction sur le skieur.
\item Trouvez le vecteur $\vec{F}_R$ qui repr�sente la force r�sultante sur le skieur.
\item Montrez par un calcul (arrondi � 4 d�cimales) que la force r�sultante est parall�le � la pente.
\item Trouvez l'acc�l�ration (scalaire) du skieur.
\end{enumerate}

\textit{Vous pouvez travailler en �quipes de deux. R�pondez � la main sur des feuilles lign�es. Attention � la clart�, � la propret� et � l'uniformit� de la pr�sentation. Montrez tous les calculs en d�tail, en particulier ceux qui impliquent des op�rations vectorielles comme des produits scalaires, des normes et des projections. Portez une attention sp�ciale � la notation math�matique. N'oubliez pas d'�crire vos noms en haut de la premi�re page et de brocher ensemble les feuilles du devoir!}

%\newpage

%{\bfseries\large R�ponses}


\end{document}